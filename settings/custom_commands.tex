%-------------------------------------------
%---------------GENERAL---------------------
%-------------------------------------------

% Phantom punctuation chars for ATS parsability
% Used in \myentry, \mysubentry and \myproject
\newcommand{\mydot}{\makebox[0pt]{\texttransparent{0}{.}}}
\newcommand{\mycomma}{\makebox[0pt]{\texttransparent{0}{,}}}

% Custom indent size, to be used with long paragraphs 
\newlength\myindentsize
\setlength{\myindentsize}{15pt} % 15pt at 10pt text size is the LaTeX standard
\newcommand{\myindent}{\makebox[\myindentsize]{}}

% Line Width 
\newlength\mylinewidth
\setlength{\mylinewidth}{\textwidth}
\addtolength{\mylinewidth}{-2em} % Here adjust margins of \myminipage as \textwidth minus em units

% Section Margins and Alignment using Minipage
\newcommand{\myminipage}[1]{%
	\makebox[\textwidth][c]{% Centered Minipage
		\begin{minipage}{\mylinewidth}% Width of Minipage
		#1
		\end{minipage}%
	}%
}

%-------------------------------------------
%---------------TABULARS--------------------
%-------------------------------------------

% CV entries
% Usage: \myentry{university-or-company}{city}{degree-or-position}{year}
\newcommand{\myentry}[4]{%
	\begin{tabularx}{\textwidth}{ l X r }
		 \textbf{#1} & \hspace*{\fill} \mycomma \hspace*{\fill} & \textit{#2}\mycomma \\
		  \textit{#3} & \hspace*{\fill} \mycomma \hspace*{\fill} & \textit{#4}\mydot
	\end{tabularx}%
}

% CV subentries, for changes of position at same company 
% Usage: \myentry{degree-or-position}{year}
\newcommand{\mysubentry}[2]{%
	\begin{tabularx}{\textwidth}{ l X r }
		  \textit{#1} & \hspace*{\fill} \mycomma \hspace*{\fill} & \textit{#2}\mydot
	\end{tabularx}%
}

% Projects
% Usage: \myproject{name}{skills-or-tooling-used}{year}
\newcommand{\myproject}[3]{%
	\begin{tabularx}{\textwidth}{ l l X r }
		  \textbf{#1} & ~$|$~\textit{#2} & \hspace*{\fill} \mycomma \hspace*{\fill} & \textit{#3}\mydot
	\end{tabularx}%
}

%-------------------------------------------
%---------------ITEMIZE---------------------
%-------------------------------------------

% Plain lists, no bullets, no headings
\newcommand{\myplainlist}[1]{%
	\begin{itemize}[
		leftmargin=1em,
		itemindent=-1em,
		label={}, 
		nosep,
		%after=\vspace{-7pt}
	]
    		#1
	\end{itemize}%
}

% Bulleted list, to be used with after \myentry or \mysubentry
% Custom bullets (2 choices)
\newcommand{\mybulleta}{$\vcenter{\hbox{\tiny$\bullet$}}$}
%
\newcommand{\mybulletb}{$\boldsymbol{\cdot}$}
%
\newcommand{\myitemize}[1]{%
  \begin{itemize}[
    leftmargin=2em,
    %itemindent=-1em,
    label={\mybulleta}, 
    nosep,
    %before=\vspace{-1\medskipamount}
    %before=\vspace{-1\baselineskip}
    ]
    \vspace{0.5ex}
    #1
  \end{itemize}%
}