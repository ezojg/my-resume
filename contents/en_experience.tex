\section{Experience}

\myminipage{
%	\myentry{Outlier AI}{Mexico (Remote)}%
%		{Bilingual AI Evaluator}{2024--2025}
%		\myitemize{
%			\item Annotated and evaluated LLM-generated conversational data in Spanish and English, ensuring quality through linguistic and semantic criteria to ensure high-quality outputs.
%			\item Contributed to model training and fine-tuning through structured feedback mechanisms, directly improving internal quality metrics for text generation and summarization tasks.
%			\item Participated in cross-task quality reviews, identifying recurrent error patterns and proposing enhancements to annotation guidelines to optimize evaluation processes.
%	}

\myentry{NeTS (IUSS Pavia)}{Pavia, Italy}%
	{NLP Research Intern}{2023--2024}
	\myitemize{
		\item Developed a novel parsing model for complex adjunction phenomena by integrating computational frameworks with contemporary cognitive models of language processing (discontinuous parsing, lexical access, top-down grammars).
		%\item Collaborated with international researchers to present parsing innovations at the \textit{31st Colloquium on Generative Grammar} (Palma, Spain), contributing to cross-disciplinary NLP methodology discussions and advancing the undestanding of complex adjuntion structures.
		\item Collaborated with international researchers to present parsing innovations at the \textit{31st Colloquium on Generative Grammar} (Palma, Spain), contributing to cross-disciplinary NLP discussions and advancing the understanding of complex adjunction structures.
}
\vspace{1ex}

\myentry{Center for Genomics Sciences (CCG-UNAM)}{Cuernavaca, Mexico}%
	{NLP Specialist}{2016--2019}
\myitemize{
	\item Provided NLP solutions for RegulonDB, a database of \textit{E.~coli} K-12 genetic regulation knowledge, developing customized text-mining methods for extracting gene-regulation data from biomedical literature databases (automatic biocuration, Python, Linux/Bash, OpenIE).	
	\item Developed hybrid text-mining components combining linguistic rule-based analysis (text simplification, thematic role identification) with machine learning (NER, relation extraction), achieving an 0.84 F-score for identifying gene regulatory pairs in biomedical literature, outperforming comparable state-of-the-art systems.
	\item The contributions made to this project enabled cross-domain adaptation of biocuration pipelines (from \textit{E. coli} to \textit{Salmonella} literature), demonstrating transferability of the text-mining framework.
	\item Linked computational linguistics and genomics teams through technical mentoring and interdisciplinary talks. %presentations
}
\vspace{1ex}

\mysubentry{NLP Research Intern}{2016}
\myitemize{
    \item Provided linguistic consultancy and contributed to modifications in the design of an automatic biocuration pipeline as part of a newly formed NLP division (text structure analysis, evaluation of text simplification methods).
    \item Prototyped initial text simplification components that were later expanded into production systems (Python, Bash). %pipeline components
}
\vspace{1ex}

\myentry{Semiosfera Marketing}{Mexico City, Mexico}%
    {Linguistic Consultant}{2016}
\myitemize{
    \item Advised on the design and curation of dialectal Mexican Spanish corpora from field interviews, establishing annotation guidelines and linguistic analysis protocols.
    \item Participated in developing annotated multimodal corpora, including transcription, analysis, and annotation (discourse analysis, prosodic/phonetic annotation, pragmatic tagging, Praat), to enable NLP applications in marketing research.
}
\vspace{1ex}
		
%	\myentry{Laboratory of Phonetic Studies (El Colegio de México)}{Mexico City, Mexico}%
%		{Field Linguist / Speech Data Specialist}{2014}
%	\myitemize{
%		\item Conducted systematic field recordings of Mexican Spanish dialects with phonetic/sociolinguistic controls, travelling across the country and  producing speech corpora for prosodic analysis.
%		%\item Applied rigorous data collection protocols (recording conditions, metadata tagging) to ensure corpus interoperability for potential speech technology applications.
%	}% prevent extra space after last itemize


\myentry{Laboratory of Phonetic Studies (El Colegio de México)}{Mexico City, Mexico}%
    {Field Linguist $|$ Speech Data Specialist}{2014}
\myitemize{
    \item Conducted systematic field recordings of dialectal Mexican Spanish with phonetic/sociolinguistic controls across multiple regions, producing speech corpora for prosodic and variationist analysis.
}% prevent extra space after last itemize
}
