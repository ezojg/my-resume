
%-----------EXPERIENCE-----------
\section{Experiencia de trabajo}

\myminipage{

\myentry{NeTS (IUSS Pavia)}{Pavia, Italia}%
	{Investigador en NLP (Pasante)}{2023--2024}
	\myitemize{
	%		\item Developed a novel parsing model for complex adjunction phenomena by integrating computational frameworks with contemporary cognitive models of language processing (discontinuous parsing, lexical access, top-down grammars).
	%  marcos computacionales con modelos cognitivos contemporáneos del procesamiento del lenguaje
		\item Diseñé un modelo experimental de parsing para fenómenos de adjunción complejos, integrando frameworks computacionales con modelos cognitivos modernos sobre el procesamiento del lenguaje (discontinuous parsing, teoría del acceso léxico, gramáticas generativas top-down).
		\item Colaboré con investigadores internacionales para presentar innovaciones en parsing en el \textit{31st Colloquium on Generative Grammar} (Palma, España), contribuyendo a discusiones sobre metodologías interdisciplinarias en NLP y al avance en la comprensión de estructuras complejas de adjunción.
		%Colaboré con investigadores internacionales para presentar innovaciones en análisis sintáctico en el		
}
\vspace{1ex}
	
\myentry{Centro de Ciencias Genómicas (CCG-UNAM)}{Cuernavaca, México}%
	{Especialista en NLP}{2016--2019}
\myitemize{
	\item Implementé soluciones en NLP para RegulonDB, una base de datos de regulación genética en \textit{E.~coli} K-12, desarrollando métodos de minería de texto especializados para extraer datos de regulación genética de literatura biomédica (biocuración automática, Python, Linux/Bash, OpenIE).
	\item Diseñé componentes híbridos de minería de texto combinando análisis basado en reglas lingüísticas (simplificación de texto, identificación de roles temáticos) con machine learning (reconocimiento de entidades, extracción de relaciones), logrando un F-score de 0.84 en identificación de pares regulatorios, superando a los sistemas existentes comparables.
	\item Dichas contribuciones permitieron la adaptación transversal de pipelines de biocuración (de literatura sobre \textit{E. coli} a \textit{Salmonella}), demostrando la transferibilidad de los métodos desarrollados.
	\item Establecí vínculos entre los equipos de lingüística computacional y de genómica mediante mentoría técnica y presentaciones interdisciplinarias regulares. %pláticas
}
\vspace{1ex}
	
\mysubentry{Investigador en NLP (Pasante)}{2016}
\myitemize{
    \item Brindé asesoría lingüística y contribuí al rediseño de un pipeline de biocuración automática en la recién formada división de NLP (análisis de estructura sintáctica, evaluación de métodos de simplificación de texto).
    \item Desarrollé prototipos de componentes para simplificación de texto que posteriormente se escalaron a sistemas de producción (Python, Bash). %componentes del pipeline
}
\vspace{1ex}
	
\myentry{Semiosfera Marketing}{Ciudad de México, México}%
    {Consultor Lingüístico}{2016}
\myitemize{
    \item Asesoré sobre el diseño y la curaduría de corpus anotados de español mexicano con datos recopilados mediante entrevistas de campo orales, estableciendo lineamientos de anotación y análisis lingüístico.
    \item Colaboré en el desarrollo de corpus multimodales anotados (transcripción, análisis del discurso, anotación fonética y prosódica, etiquetado pragmático, Praat) para aplicaciones de NLP en investigación de mercado.
}
\vspace{1ex}

\myentry{Laboratorio de Estudios Fónicos (El Colegio de México)}{Ciudad de México, México}%
    {Lingüista de campo $|$ Especialista en datos de habla}{2014}
\myitemize{
    \item Realicé grabaciones de campo sistemáticas de español mexicano dialectal con controles fonéticos y sociolingüísticos en múltiples regiones, generando corpus de habla para el análisis prosódico y variacionista.
}% prevent extra space after last itemize
}